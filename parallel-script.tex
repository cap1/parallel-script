\documentclass{article}
\usepackage[utf8]{inputenc}
\usepackage{graphicx}
\usepackage{amsmath}
\usepackage{booktabs}
\usepackage{textcomp}
\usepackage{multirow}
\usepackage{color}

\author{Christian Mueller}
\title{Parallel Computing Script}

\begin{document}

\maketitle
\tableofcontents

\section{Introduction} % (fold)
\label{sec:introduction}
	The computing power has been growing ever since the first integrated circuit,
	but it has an ultimate limit: the speed of light.
	The performance of microprocessors has been increasing by 50\% each year from 1986 till 2002.
	Since then the rate has decreased to only about 20\% per year.
	This gain in performance resulted in the reduction of the size of integrated circuits.
	But as ICs become more dense for faster processing,
	they also consume more power and therefore generate more unwanted heat.
	To overcome the need for higher clock-rates,
	whom are needed to increase the performance on single core processing units,
	one is using multiple cores instead.
	With this parallelism, computers can gain in performance.
% section introduction (e


\end{document}